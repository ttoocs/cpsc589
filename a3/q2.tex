(20 marks) A closed mesh (i.e. no boundary vertices) is saved in a half-edge data structure. For a given face f , write, in pseudocode, an algorithm that

\begin{enumerate}
\item finds all faces adjacent to f \\
\begin{verbatim}
let f be our current face,
e = f.edge;
adj<faces>;
do
  adj.push_back( e.pair.face );
  e=e.next;
while(e != f.getEdge);

return adj;
\end{verbatim}
This simply iterates of each edge of the face, and appends it's face to the returned value.\\
\item all vertices that are on f or connected to f by an edge. \\
\begin{verbatim}
let f be our current face,
e = f.edge;
verts<verticies>;

do
  verts.push_back(e.source);
  e2 = e;
  do
    verts.push_back(e2.source);
    e2 = e2.next.pair
  while ( e2 != e || e2 != e.next.pair)
  e=e.next;
while(e != f.getEdge);

return verts;
\end{verbatim}
Outer loop, iterates over all all vertces that are on f, and adds them. \\
Inner loop, iterates over all edges of a given vertex, and adds their verticies (excet the one we're currently one, and it's next one, as that'll get added in the next iteration of the outer loop)
\end{enumerate}
