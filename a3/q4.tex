\begin{itemize}
\item Doo-Sabin method: \\
From the origonal paper: \\
"In this process, 3 types of new faces can be formed: \\
a) Type F: A 5-sided face will give a new smaller 5-sided face within itself and bears a smiliar shape, this type of new face is termed Type F (fromed by face). \\
b) Type V: A vecterx common to 4 faces, i.e. a corner where 3 faces joined togeather having three common boundaries, will produce a 3-sided face, this is Termed type V, (fromed by vertex). \\
c) Type E: On each common boundary of two adjacent faces, a 4-sided face will be formed, this is termed Type E (fromed by edge). \\

  The new polyhedron will consist of these 3 types of new faces. A n-sided face will provide a basis for a smaller n-sided F type new face, it will remain n-sided as the subdivision carries on and will gradually converge to the centroid and diminish to an acceptable size. A common edge will always produce a 4-sided new face, and m-spoked vertex will produce a m-sided V type face, which will, in turn, become the basis of a smaller m-sided F type face in the next subdivision process." \\

Taking this togeather, you have:
$f' = f + e + v$ \\

(Again from the paper, as I have read it and cannot un-read it)
Each e will give a 4-sided polygon with 4 new verticies, however any two adjacent edges on a face will generate only one new vertex on that face, therefore the number of new verticies formed will be: \\
$v' = 4 e / 2 = 2e$ \\
Note: This can also be verified by the expansion/shrinking method of thinking about it: As each edge is shrunk/extended, the end of each old-edge, now creates a vertex, independed of any other edge. This gives $v' = 2e$. \\

Finnally, we use Euler's formula for polyhedron: $F+V-E=2$ to get the number of edges: \\
$e'=f'+v'-2$ \\
$e'=f+e+v + 2e + -2$ \\
$e'= f + v + 3e -2$ \\


\item Loop method: \\
Each edge is divided in half, giving  \\
$e' = 2e + (e/3)*3 = 2e+ e = 3e;$ \\
$v' = v+e;$ \\
and $f' = f*4$ \\
%so using euler's formula to verify on a single triangle: \\
%$F+V-E=2$ \\
%Single triangle: e=3; f=1, v=3;\\
%1+3-3=2
\end{itemize}
